\documentclass[11pt]{article}
\usepackage{amsmath,euscript}
\usepackage{amssymb}
\usepackage{amsthm}
\usepackage{enumerate}
\usepackage{amsfonts}
\usepackage{amscd}
\usepackage{courier}
\usepackage{verbatim}
\usepackage{fancyvrb}

\begin{document}

\title{Project 1: Root Finding via Lagrange Polynomial Interpolation}
\date{\today}
\author{Max Ortman \and Teammate 1 \and Teammate 2} 

\maketitle

\section*{The Algorithm}

Below is our implementation of the root-finding algorithm using Lagrange Polynomials to estimate the zero of a function.

\begin{Verbatim}
# [Insert code here]
\end{Verbatim}

\section*{Explanation}

To estimate the zero for this function, we use sampled points to construct a Lagrange Polynomial. The basis polynomials are defined as:

$$ L_i(x) = \prod_{j \ne i} \frac{(x - x_j)}{(x_i - x_j)} $$

Using these basis functions, the unique polynomial of degree $n$ that passes through all $n+1$ points is:

$$ P(x) = \sum_i y_i L_i(x) $$

To start our method we sample the beginning, end, and middle $n-1$ points. This gives us a set of $n+1$ points to construct our Lagrange Polynomial.
We then evaluate $P(x)=0$ to get our initial estimate of the root. We then add that as a middle point and remove the point with the largest $x$ value while making sure there is at least one point on each side of zero. This process is repeated until we have reached the desired number of iterations or the error is sufficiently small.

\section*{Analysis \& Results}


\vspace{5mm}
\begin{tabular}{ll}
iter & average error \\
\hline
% [Insert data here]
\end{tabular}
\vspace{5mm}

\section*{Order of Convergence}

IDK this part cus we don't know how many points we will use or if we can figure it out. I think we have a good enough/creative enough that we dont need this part

\end{document}
